%%%%%%%%%%%%%%%%%%%%%%%%%%%%%%%%%%%%%%%%%%%%%%%%%%%%%%%%%%%
% EPFL report package, main thesis file
% Goal: provide formatting for theses and project reports
% Author: Mathias Payer <mathias.payer@epfl.ch>
%
% This work may be distributed and/or modified under the
% conditions of the LaTeX Project Public License, either version 1.3
% of this license or (at your option) any later version.
% The latest version of this license is in
%   http://www.latex-project.org/lppl.txt
%
%%%%%%%%%%%%%%%%%%%%%%%%%%%%%%%%%%%%%%%%%%%%%%%%%%%%%%%%%%%
\documentclass[a4paper,12pt,oneside]{report}
% Options: MScThesis, BScThesis, MScProject, BScProject
\usepackage[BScThesis]{EPFLreport}
\usepackage{xspace}
\usepackage{caption}
\usepackage{subfig}
\usepackage[font=small,labelfont=bf]{caption}

\onehalfspacing

\title{Rotation estimation of detected satellites}
\author{Noah Kaltenrieder (301368)}
\supervisor{Mathieu Salzmann}
%\adviser{Prof. Dr. sc. ETH Mathias Payer}
%\coadviser{Second Adviser}
%\expert{The External Reviewer}

\newcommand{\sysname}{FooSystem\xspace}

\begin{document}
\maketitle
%\makededication
\acknowledgments{
    I really would like to thank my supervisor Prof. Mathieu Salzmann who helped me a lot and all the people that participated
    in a way to this project, namely Dr. Cameron Lemon, Prof. Jean-Paul Kneib, Prof. Frédéric Courbin, and the SSA Team. I also want to express
    my gratitude to Yann Bouquet who helped me with pleasure when I had questions. Finally, I also would like to especially thank 
    Alexandre Di Piazza and Marcellin Feasson with whom I have collaborated through all the semester in different ways.
    \newline 
    Thanks a lot to Prof. Mathias Payer for the thesis template that I used to write this report~\cite{templateRef}
}
\makeacks

\begin{abstract}
We present an approach to extract the intensity along the satellite streaks that can be found in images that were obtained by OMEGACAM 
on the VLT Survey Telescope. In this project, we only analyzed the long tracks that were created by high velocity objects, to
try to get an estimation about their rotation. We used Fourier transform to get the periodicity of the intensity along the tracks.
Due to the lack of real data about rotation, we first had to create fake streaks with sinusoid intensity and different angles.
We manually got the tracks to run the code on them, so ideally this project should be integrated with DetectSat~\cite{detectsatRepository} so that all 
the process can be done automatically. We found very great result with it and run the pipeline on real tracks. By making assumptions about the satellite, 
like their altitude, we get possible results on real streaks, but since we don't have access to the real values, we can't say if it is close to reality or not.
\end{abstract}

%\begin{frenchabstract}
%For a doctoral thesis, you have to provide a French translation of the
%English abstract. For other projects this is optional.
%\end{frenchabstract}

\maketoc

%%%%%%%%%%%%%%%%%%%%%%
\chapter{Introduction}
%%%%%%%%%%%%%%%%%%%%%%

This project is about estimating the rotation of spatial objects detected by the OMEGACAM on the VLT Survey Telescope. In these times,
we have a lot of flying objects, it can be debris, useful satellites or dead satellites, around the Earth and it becomes more and more
necessary to have a clear view of all these objects.
The project of Yann Bouquet "DetectSat"\cite{detectsatRepository} was focused on detection of the satellite tracks on fits files, but with both,
small (low velocity) and long (high velocity) streaks. For this project we had to focus on the long tracks, so that the variation of intensity
along the line could be better detected. The rotation estimation can be used to determine which kind of object the streak is and, if it is
a satellite, it may be used to better find out which one it is. Ideally, this should be run on the lines detected by DetectSat so all the process
could be done with only the fits file as input.
\newline
\newline
The main challenge is that we don't have access to data about rotation of satellites, so we had to first create our own streaks 
with sinusoid intensity. This lack of resources is very restrictive, since even if we get a result, we cannot know the precision of this method 
and we cannot be sure that it is right or completely wrong. This is why we had to begin with creating lines that seem realistic and
try with random angles.
\newline
\newline
The second part of this project was to retrieve the periodicity of the sinusoidal function along the streaks that we have previously generated, 
and finally try this on real tracks. Since this project does not recognize the lines, we had to find the coordinates of the streaks by hand and give them
as input to the pipeline. For this part, there are many factors that can decrease the precision of the pipeline, e.g., if the line crosses a
really bright star, it gives a peak of intensity and can lead to wrong result, or if the brightness of the streak is too small
compared to the brightness of the background of the sky on the image. 


%%%%%%%%%%%%%%%%%%%%
\chapter{Inputs}
%%%%%%%%%%%%%%%%%%%%

In this project, we used fits images that were generated by the OMEGACAM on the VLT Survey Telescope, and we used a script to create a 
mosaic of $32$ images, separated by NaN values, that are $4'000$$\times$$2'000$ pixels. So the image with all the mosaic's dimension is $16'000$$\times$$16'000$ pixels. 
A mosaic is approximately $1.5$-$2$ GB of data. 
\newline
We principally used the fits file "\emph{OMEGA.2020-02-22T02:11:36.295\_fullfield.fits}", corresponding to the
captured image February 22, 2020, at 02:11:36, and with the pixels scaled using \textbf{ZScale} to be able to see the streaks as wanted.
\newline
\begin{center}
    \includegraphics{omega_36.295_report}
    \captionof{figure}{The main FITS image that we used}
    \label{mainFits}
\end{center}
The pipeline is being run on the entire file, with the coordinates that are relative to the entire file : the origin is at the bottom left of
the fits file. The first part of this project, which is about line generation, is completely based on this file Figure \ref{mainFits}. We chose
this file because we can clearly see a really interesting track that goes through the image :
\begin{center}
    \includegraphics[scale=0.06]{omega_36.295_streak_report}
    \captionof{figure}{The interesting streak}
    \label{mainFitsStreak}
\end{center}
This track on Figure \ref{mainFitsStreak}, the one that goes through four blocks, was clearly visible and the variation of intensity can also be notified, 
so it was a great way to test the pipeline with a streak that is not too difficult. Thanks to Marcellin, he could identify this and it turned out that 
this streak is actually a debris of the European carrier rocket Ariane 2 ! We know that it is at a geostationary altitude, 
so $\sim 36'000\ [km]$ and has a speed of $\sim 3'065\ [\frac{m}{s}]$. We will use this information later to analyze the coherence 
of the result obtained by the pipeline.

%%%%%%%%%%%%%%%%
\chapter{Line Generation}
%%%%%%%%%%%%%%%%

When we started this project, we had no data about rotation of satellites and their corresponding tracks on a file, so we had to first 
generate streaks on the image myself, to be able to test the pipeline. First, to generate a line we had to choose an existing star in the file
and cut it out to have a PSF to convolve my line with. The PSF : "Point Spread Function" is like the quantity of blurring of a point object.
Here, we had to use it so that my generated line seems more realistic in the image. 

%%Explain the convolution


\begin{figure}[h]
    \begin{minipage}[c]{.5\linewidth}
        \centering
        \includegraphics[scale=0.1]{report_line_not_convolved}
        \caption{The line without using the PSF}
    \end{minipage}
    \hfill
    \begin{minipage}[c]{.5\linewidth}
        \centering
        \includegraphics[scale=0.1]{report_line_convolved}
        \caption{The line using the PSF}
    \end{minipage}
\end{figure}

We will now explain the method we have used to achieve this result. First, we have to choose a star by hand and cut it out to use it as a PSF.
The dimension of the star cut out is $30$px, so we initialized an array of height's dimension of $30$px and an arbitrary width, with zeros everywhere,
except at the center of the array, the center line of one pixel of height at the center is set with sinusoid values. Once, we have these two pieces,
we can convolve the two arrays together to make the final line with an appropriate PSF. 
\begin{figure}[h]
    \begin{minipage}[c]{.47\linewidth}
        \centering
        \includegraphics[width=\linewidth, scale=0.5]{line_before_convolution}
        \caption{Intensity of each row of the line's array}
    \end{minipage}
    \hfill
    \begin{minipage}[c]{.47\linewidth}
        \centering
        \includegraphics[width=\linewidth, scale=0.5]{line_after_convolution}
        \caption{Intensity of each row of the line's array after the convolution}
    \end{minipage}
\end{figure}
\newline
We finally rotated the final convolved line by a random angle and add it to the image by summing up the values with the image's array. 
\begin{center}
    \includegraphics[scale=0.07]{report_image_line_rotated}
    \captionof{figure}{The generated line on the image}
    \label{lineOnImage}
\end{center}
The final result of the line in the image is pretty satisfying and coherent, as we can see on Figure \ref{lineOnImage}.
\newline
The inconvenience of this method is that we had to choose a star by hand and it will work great for this specific file, if we want to make this code 
works with another image we have to choose a star from the new file to have a coherent PSF to integrate the line better on it.


%%%%%%%%%%%%%%%%%%%%%%%%
\chapter{Find the intensity along the track}
%%%%%%%%%%%%%%%%%%%%%%%%

Once the generation part was done, we started the main part of the project that was finding back the periodicity of the previously generated sinusoid 
intensity streak and finally apply the final pipeline on real tracks.
\newline
\newline
The method that we used was very simple : We just sum up all the values on the $y$ coordinate for all $x$ coordinate, so it returns an array 
with a height of one and with a width as long as the streak. Here, the width was found manually, since we had to enter the start and the end coordinates 
of the track as input for the pipeline, it was trivial to recover the width of the line, even when the line was not horizontal, by using trigonometry. 
\begin{center}
    $ \displaystyle width\_of\_the\_line = \frac{end\_line\_x - start\_line\_x}{cos(angle\_line\_in\_radian)}$
\end{center}
Once we have this array with the intensity of each $x$ point, we can consider it as a signal and use the fast Fourier transform (FFT) algorithm \cite{fastFourierTransform} 
on the array to calculate the discrete Fourier transform (DFT) that can be used to find the frequency of the signal. We also used the inverse FFT
to try to remove a bit the noise that we get from the image; indeed the line is not isolated from the background of other stars or other streaks 
that can add more or less noise to the signal. We also choose to use a threshold for the number of periods that can be in a track to remove a lot of noise too. 
We have set it arbitrarily to $100$ repetitions, because the width of a streak is limited to the width of one image of the mosaic, which is 
$2'000$px. It would mean that we would see a repetition every $20$px, so the frequency of the signal would be $0.05\ [$px$^{-1}]$ and,
assuming a geostationary object, this would imply $4.18$ rotation per second and it is very unlikely that an object has such a high 
rotation rate.
\newline
\newline
Finally, we used the "\textbf{curve\_fit}" function from the scipy library that use the least square optimization to find the frequency of the
original sinusoid function. But to use this function, we have to make a good first guess of the parameters, so that they will converge 
fast into the right value. A good first guess for the frequency, which is the only parameter of the sinusoid function that really interests us, is to 
divide the number of cycles that we found with the Fourier analysis part, by the width of the streak. 
\newline
Using this pipeline, we usually find good estimation of the wave but it can vary a lot when there are a lot of noises.
\newline
\newline
We cannot use the result directly found with the pipeline explained just above, because it expresses the frequency by pixel, to be able to
compare it with a reference value as we would like, we have to convert it to spatial scale e.g., for the frequency $[s^{-1}]$. To do this
we had to make some assumptions : 
\begin{enumerate}
    \item The flying object is geostationary
    \begin{itemize}
    \item So the altitude would be $\sim 36'000'000\ [m]$
    \item With a speed of $3'065\ [\frac{m}{s}]$
    \end{itemize}
    \item The fits file has a scale of $0.21\ [\frac{arcsec}{pixel}]$
\end{enumerate}
With these elements, we can find the rotation rate easily :
\begin{center}
$one\_cycle\_in\_pixel = \frac{1}{frequency\_found}\ [pixel]$

$arcsec\_by\_cycle = 0.21 \times one\_cycle\_in\_pixel\ [arcsec]$

$distance\_one\_cycle = 36'000'000 \times (\frac{\pi}{180} \times \frac{arcsec\_by\_cycle}{3'600})\ [m]$

$rotation\_rate = \frac{3'065}{distance\_one\_cycle}\ [s^{-1}]$
\end{center}
This rotation rate, which is the number of rotation of the object in one second, can be used to compare what we have found with 
the existing information about the objects that we will detect.

\section{Result on Simulated data}
Running the pipeline on a line that we created previously gives pretty satisfying results. Here we can see the result with a basic case with an horizontal line 
but that does cross a star or bright object.
\begin{figure}[h]
    \begin{minipage}[c]{.47\linewidth}
        \centering
        \includegraphics[width=\linewidth, scale=0.5]{fit_find_line_horizontal}
        \caption{Total intensity by x coordinate on the line}
        \label{graphHorizontalLine}
    \end{minipage}
    \hfill
    \begin{minipage}[c]{.47\linewidth}
        \centering
        \includegraphics[width=\linewidth, scale=0.5]{report_crossed_line}
        \caption{The star that is crossed by the line}
    \end{minipage}
\end{figure}

As we can see on Figure \ref{graphHorizontalLine}, the fitted function seems really close to the data even if there is a peak of intensity
at some point, because of the star crossed. It is a good sign that our pipeline can resist, even if it is a small amount of noise here.

%%%%%%%%%%%%%%%%%%%%
\chapter{Cast the frequency}
%%%%%%%%%%%%%%%%%%%%

We cannot use the result directly found with the pipeline explained just above, because it expresses the frequency by pixel, to be able to
compare it with a reference value as we would like, we have to convert it to spatial scale e.g., for the frequency $[s^{-1}]$.
\newline



%%%%%%%%%%%%%%%%%%%%%%
\chapter{Final Result}
%%%%%%%%%%%%%%%%%%%%%%


Running the pipeline on a line that we created previously, takes approximately less than 10 seconds, by block of the mosaic, using a 1,8 GHz Intel Core i5 dual core
on a MacBook Air 2017. 


The related work section covers closely related work. Here you can highlight
the related work, how it solved the problem, and why it solved a different
problem. Do not play down the importance of related work, all of these
systems have been published and evaluated! Say what is different and how
you overcome some of the weaknesses of related work by discussing the 
trade-offs. Stay positive!

This section is usually 3-5 pages.


%%%%%%%%%%%%%%%%%%%%
\chapter{Conclusion}
%%%%%%%%%%%%%%%%%%%%

In the conclusion you repeat the main result and finalize the discussion of
your project. Mention the core results and why as well as how your system
advances the status quo.

\cleardoublepage
\phantomsection
\addcontentsline{toc}{chapter}{Bibliography}
\printbibliography

% Appendices are optional
% \appendix
% %%%%%%%%%%%%%%%%%%%%%%%%%%%%%%%%%%%%%%
% \chapter{How to make a transmogrifier}
% %%%%%%%%%%%%%%%%%%%%%%%%%%%%%%%%%%%%%%
%
% In case you ever need an (optional) appendix.
%
% You need the following items:
% \begin{itemize}
% \item A box
% \item Crayons
% \item A self-aware 5-year old
% \end{itemize}

\end{document}