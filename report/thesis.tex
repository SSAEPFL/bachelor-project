%%%%%%%%%%%%%%%%%%%%%%%%%%%%%%%%%%%%%%%%%%%%%%%%%%%%%%%%%%%
% EPFL report package, main thesis file
% Goal: provide formatting for theses and project reports
% Author: Mathias Payer <mathias.payer@epfl.ch>
%
% This work may be distributed and/or modified under the
% conditions of the LaTeX Project Public License, either version 1.3
% of this license or (at your option) any later version.
% The latest version of this license is in
%   http://www.latex-project.org/lppl.txt
%
%%%%%%%%%%%%%%%%%%%%%%%%%%%%%%%%%%%%%%%%%%%%%%%%%%%%%%%%%%%
\documentclass[a4paper,12pt,oneside]{report}
% Options: MScThesis, BScThesis, MScProject, BScProject
\usepackage[BScThesis]{EPFLreport}
\usepackage{xspace}
\usepackage{caption}

\onehalfspacing

\title{Rotation estimation of detected satellites}
\author{Noah Kaltenrieder (301368)}
\supervisor{Mathieu Salzmann}
%\adviser{Prof. Dr. sc. ETH Mathias Payer}
%\coadviser{Second Adviser}
%\expert{The External Reviewer}

\newcommand{\sysname}{FooSystem\xspace}

\begin{document}
\maketitle
%\makededication
\acknowledgments{
    I really would like to thank my supervisor Prof. Mathieu Salzmann who helped me a lot and all the people that participated
    in a way to this project, namely Dr. Cameron Lemon, Prof. Jean-Paul Kneib, Prof. Frédéric Courbin, and the SSA Team. I also want to express
    my gratitude to Yann Bouquet who helped me with pleasure when I had questions. Finally, I also would like to especially thank 
    Alexandre Di Piazza and Marcellin Feasson with whom I have collaborated through all the semester in different ways.
    \newline 
    Thanks a lot to Prof. Mathias Payer for the thesis template that I used to write this report~\cite{templateRef}
}
\makeacks

\begin{abstract}
I present an approach to extract the intensity along the satellite streaks that can be found in images that were obtained by OMEGACAM 
on the VLT Survey Telescope. In this project, I only analyzed the long tracks that were created by high velocity objects, to
try to get an estimation about their rotation. I used Fourier transform to get the periodicity of the intensity along the tracks.
Due to the lack of real data about rotation, I first had to create fake streaks with sinusoid intensity and different angles.
I manually get the tracks to run my code on them, so ideally this project should be integrated with DetectSat~\cite{detectsatRepository} so that all 
the process can be done automatically. I found very great result with it and run the pipeline on real tracks. By making assumptions about the satellite, 
like their altitude, I get possible results, but since we don't have access to the real values, we can't say if it is close to reality or not.
\end{abstract}

%\begin{frenchabstract}
%For a doctoral thesis, you have to provide a French translation of the
%English abstract. For other projects this is optional.
%\end{frenchabstract}

\maketoc

%%%%%%%%%%%%%%%%%%%%%%
\chapter{Introduction}
%%%%%%%%%%%%%%%%%%%%%%

This project is about estimating the rotation of spatial objects detected by the OMEGACAM on the VLT Survey Telescope. In these times,
we have a lot of flying objects, it can be debris, useful satellites or dead satellites, around the Earth and it becomes more and more
necessary to have a clear view of all these objects.
The project of Yann Bouquet "DetectSat"\cite{detectsatRepository} was focused on detection of the satellite tracks on fits files, but with both,
small (low velocity) and long (high velocity) streaks. To this project I had to focus on the long tracks, so that the variation of intensity
along the line could be better detected. The rotation estimation can be used to determine which kind of object the streak is and, if it is
a satellite, it may be used to better find out which one it is. Ideally, this should be run on the lines detected by DetectSat so all the process
could be done with only the fits file as input.
\newline
\newline
The main challenge is that we don't have access to data about rotation of satellites, so I had to first create my own streaks 
with sinusoid intensity. This lack of resources is very restrictive, since even if I get a result, we cannot know the precision of this method 
and we cannot be sure that it is right or completely wrong. This is why I had to begin with creating lines that seem realistic and
try with random angles.
\newline
\newline
The second part of this project was to retrieve the periodicity of the sinusoidal function along the streaks that I generated, and finally
try this on real tracks. Since my project does not recognize the lines, I had to find the coordinates of the streaks by hand and give them
as input to the pipeline. For this part, there are many factors that can decrease the precision of the pipeline, e.g., if the line crosses a
really bright star, it gives a peak of intensity and can lead to wrong result, or if the brightness of the streak is too small
compared to the brightness of the background of the sky on the image. 


%%%%%%%%%%%%%%%%%%%%
\chapter{Inputs}
%%%%%%%%%%%%%%%%%%%%

In this project, I used fits images that were generated by the OMEGACAM on the VLT Survey Telescope, and I used a script to create a 
mosaic of 32 images, separated by NaN values, that are 4000x2000 pixels. So the image with all the mosaic's dimension is 16'000x16'000 pixels. 
A mosaic is approximately 1.5-2 GB of data. 
\newline
I principally used the fits file "\emph{OMEGA.2020-02-22T02:11:36.295\_fullfield.fits}", corresponding to the
captured image February 22, 2020, at 02:11:36, and with the pixels scaled using ZScale to be able to see the streaks as wanted.
\newline
\begin{center}
    \includegraphics{omega_36.295_report}
    \captionof{figure}{The main FITS image that I used}
    \label{mainFits}
\end{center}
The pipeline is being run on the entire file, with the coordinates that are relative to the entire file : the origin is at the bottom left of
the fits file. The first part of this project, which is about line generation, is completely based on this file Figure \ref{mainFits}. I chose
this file because we can clearly see a really interesting tracks that goes through the image :
\begin{center}
    \includegraphics[scale=0.06]{omega_36.295_streak_report}
    \captionof{figure}{The interesting streak}
    \label{mainFitsStreak}
\end{center}
This track on Figure \ref{mainFitsStreak} was clearly visible and the variation of intensity can also be notified, so it was a great way to test the pipeline with a not
too tricky streak. Thanks to Marcellin, he could identify this and it turned out that this streak is actually a debris of the European 
carrier rocket Ariane 2 ! We know that it is at a geostationnary altitude, so \textasciitilde 36'000 [km] and has a speed of \textasciitilde 3065 [m/s]. I'll use these
informations later to analyse the coherence of the result obtained by the pipeline.

%%%%%%%%%%%%%%%%
\chapter{Line generation}
%%%%%%%%%%%%%%%%

When I started this project, we had no data about rotation of satellites and their corresponding tracks on a file, so I had to first 
generate streaks on the image myself, to be able to test the pipeline. First, to generate a line I had to chose a existing star in the file
and cut it out to have a PSF to convolve my line with. The PSF : "Point Spread Function", is like the quantity of blurring of a point object.
Here, I had to use it so that my generated line seems more realistic in the image. 

\begin{center}
    \includegraphics[scale=0.1]{report_line_not_convolved}
    \captionof{figure}{The line without using the PSF}
    \label{lineNotConvolved}
    \includegraphics[scale=0.1]{report_line_convolved}
    \captionof{figure}{The line using the PSF}
    \label{lineConvolved}
\end{center}

I will now explain the method I have used to achieve this result. First, I have to chose a star by hand and cut it out to use it as a PSF.
The dimension of the star cutted out is 30px, so I initialized an array of height's dimension of 30px and an arbitrary width, with zeros everywhere,
except at the center of the array, the center line of 1 pixel of height at the center is set with sinusoid values. Once, I have this two pieces,
I can convolve the two arrays together to have the final line with an adequate PSF. 
\newline
The inconvenient of this method is that I had to chose a star by hand and it will work great for this specific file, if we want to make this code 
works with another image we have to chose a star from the new file to have a coherent PSF to integrate the line better on it.


%%%%%%%%%%%%%%%%%%%%%%%%
\chapter{Implementation}
%%%%%%%%%%%%%%%%%%%%%%%%

The implementation covers some of the implementation details of your project.
This is not intended to be a low level description of every line of code that
you wrote but covers the implementation aspects of the projects.

This section is usually 3-5 pages.


%%%%%%%%%%%%%%%%%%%%
\chapter{Evaluation}
%%%%%%%%%%%%%%%%%%%%

In the evaluation you convince the reader that your design works as intended.
Describe the evaluation setup, the designed experiments, and how the
experiments showcase the individual points you want to prove.

This section is usually 5-10 pages.


%%%%%%%%%%%%%%%%%%%%%%
\chapter{Related Work}
%%%%%%%%%%%%%%%%%%%%%%

The related work section covers closely related work. Here you can highlight
the related work, how it solved the problem, and why it solved a different
problem. Do not play down the importance of related work, all of these
systems have been published and evaluated! Say what is different and how
you overcome some of the weaknesses of related work by discussing the 
trade-offs. Stay positive!

This section is usually 3-5 pages.


%%%%%%%%%%%%%%%%%%%%
\chapter{Conclusion}
%%%%%%%%%%%%%%%%%%%%

In the conclusion you repeat the main result and finalize the discussion of
your project. Mention the core results and why as well as how your system
advances the status quo.

\cleardoublepage
\phantomsection
\addcontentsline{toc}{chapter}{Bibliography}
\printbibliography

% Appendices are optional
% \appendix
% %%%%%%%%%%%%%%%%%%%%%%%%%%%%%%%%%%%%%%
% \chapter{How to make a transmogrifier}
% %%%%%%%%%%%%%%%%%%%%%%%%%%%%%%%%%%%%%%
%
% In case you ever need an (optional) appendix.
%
% You need the following items:
% \begin{itemize}
% \item A box
% \item Crayons
% \item A self-aware 5-year old
% \end{itemize}

\end{document}